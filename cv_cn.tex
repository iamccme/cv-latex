%%%%%%%%%%%%%%%%%%%%%%%%%%%%%%%%%%%%%%%%%
% "ModernCV" CV and Cover Letter
% LaTeX Template
% Version 1.1 (9/12/12)
%
% This template has been downloaded from:
% http://www.LaTeXTemplates.com
%
% Original author:
% Xavier Danaux (xdanaux@gmail.com)
%
% License:
% CC BY-NC-SA 3.0 (http://creativecommons.org/licenses/by-nc-sa/3.0/)
%
% Important note:
% This template requires the moderncv.cls and .sty files to be in the same 
% directory as this .tex file. These files provide the resume style and themes 
% used for structuring the document.
%
%%%%%%%%%%%%%%%%%%%%%%%%%%%%%%%%%%%%%%%%%

%----------------------------------------------------------------------------------------
%	PACKAGES AND OTHER DOCUMENT CONFIGURATIONS
%----------------------------------------------------------------------------------------

\documentclass[ctexart,11pt,a4paper,sans]{moderncv} % Font sizes: 10, 11, or 12; paper sizes: a4paper, letterpaper, a5paper, legalpaper, executivepaper or landscape; font families: sans or roman

\moderncvstyle{casual_cn} % CV theme - options include: 'casual' (default), 'classic', 'oldstyle' and 'banking'
\moderncvcolor{blue} % CV color - options include: 'blue' (default), 'orange', 'green', 'red', 'purple', 'grey' and 'black'

\usepackage[scale=0.75]{geometry} % Reduce document margins
%\usepackage{CJKutf8}
\usepackage{xeCJK}
\usepackage{footmisc}
\usepackage[firstyear=2007, lastyear=2014]{moderntimeline}
%\setlength{\hintscolumnwidth}{3cm} % Uncomment to change the width of the dates column
%\setlength{\makecvtitlenamewidth}{10cm} % For the 'classic' style, uncomment to adjust the width of the space allocated to your name

\makeatletter
\tikzset{
    tl@startyear/.append style={
        xshift=(0.5-\tl@startfraction)*\hintscolumnwidth,
        anchor=base
    }
}
\makeatother

%----------------------------------------------------------------------------------------
%	NAME AND CONTACT INFORMATION SECTION
%----------------------------------------------------------------------------------------
\firstname{程} % Your first name
\familyname{陈} % Your last name

% All information in this block is optional, comment out any lines you don't need
\title{个人简历}
\address{北京市, 海淀区}{西土城路10号, 北京邮电大学 100876}
\mobile{(086) 15810537277}
\email{cc@iamcc.me}
\homepage{www.iamcc.me}{iamcc.me} % The first argument is the url for the clickable link, the second argument is the url displayed in the template - this allows special characters to be displayed such as the tilde in this example
%\extrainfo{additional information}
\photo[70pt][0.4pt]{pictures/jl.jpg} % The first bracket is the picture height, the second is the thickness of the frame around the picture (0pt for no frame)

%----------------------------------------------------------------------------------------

\setCJKmainfont[BoldFont=黑体, ItalicFont=楷体]{宋体} 
\begin{document}
%\begin{CJK}{UTF8}{gkai}

\makecvtitle % Print the CV title

%----------------------------------------------------------------------------------------
%	EDUCATION SECTION
%----------------------------------------------------------------------------------------

\section{教育背景}

\tllabelcventry{2011}{0}{2011.9}{工学硕士}{北京邮电大学}{}{\textit{计算机学院}}{}  % Arguments not required can be left empty
\tllabelcventry{2007}{2011}{2007.9-2011.7}{工学学士}{北京邮电大学}{}{\textit{计算机学院}}
{}

%\section{Masters Thesis}

%\cvitem{Title}{\emph{Money Is The Root Of All Evil -- Or Is It?}}
%\cvitem{Supervisors}{Professor James Smith \& Associate Professor Jane Smith}
%\cvitem{Description}{This thesis explored the idea that money has been the cause of untold %anguish and suffering in the world. I found that it has, in fact, not.}

%----------------------------------------------------------------------------------------
%	WORK EXPERIENCE SECTION
%----------------------------------------------------------------------------------------

\section{相关经验}

%------------------------------------------------

\subsection{实习经历}

\tllabelcventry{2011.5}{2012.3}{2011.5-2012.3}{实习测试开发工程师}{百度}{百度图片测试团队}{}{
\itshape
所在小组主要负责百度图片前后端引擎模块的测试。
\upshape
\begin{itemize}
\item 作为测试负责人,参与了\bfseries{百度壁纸app}\mdseries项目设计评审、测试、开发、上线
\item 开发了web自动化测试框架Nuke,基于Ruby及Watir,并在NSQA内部推广Nuke,将其与Hudson结合起来, 实现了线上前端界面的实时监控以及代码提交后的自动回归测试
\end{itemize}}

\tllabelcventry{2012.5}{2013.3}{2012.5-2013.3}{实习开发工程师}{阿里云}{阿里云搜索部门推荐团队}{}{
\itshape
所在小组主要负责阿里云搜索产品和推荐产品的数据处理和数据挖掘。
\upshape
\begin{itemize}
\item 作为项目负责人,参与了\bfseries{阿里云搜索新统计平台}\mdseries项目设计、开发
\item 作为项目负责人,参与了\bfseries{数据监控平台}\mdseries项目设计、开发
\item 开发基于飞天的Map-Reduce程序,为内部人员和站长提供数据统计
\end{itemize}}


\tllabelcventry{2013}{0}{2013.6-Present}{实习搜索研发工程师}{阿里妈妈}{阿里妈妈展示外投技术部}{}{
\itshape
所在小组主要负责阿里妈妈钻石展位的外投部分。
\upshape
\begin{itemize}
\item 接手UserInfo模块,进行组内串讲
\item 分析钻展外投定向召回率低以及竞价成功率低的问题,负责多维度定向项目的数据调研部分
\end{itemize}}

%------------------------------------------------

\subsection{学校项目}

\tllabelcventry{2012}{0}{2012.2-Present}{基于多目标优化算法的公交排班}{项目负责人}{}{}{
\itshape该项目通过调研公交排班问题的实际需求,设计相应的模型,用C++以及STL实现算法,该算法基于多目标遗传算法NSGA-II。
\upshape
\begin{itemize}
\item 作为项目负责人,设计算法实现方案,统一编码规范,推进项目执行。
\end{itemize}}

%------------------------------------------------

\tllabelcventry{2009}{2010}{2009.3-2009.6}{旅游模拟查询系统}{项目参与者}{}{}{
\itshape
城市之间有三种交通工具(汽车、火车和飞机)相连,某旅客于某一时刻向系统提出旅行要求,系统根据该旅客的要求为其设计一条旅行线路并输出。
\upshape
\begin{itemize}
\item 参与技术方案设计,负责寻找线路算法的开发,用C++语言开发
\end{itemize}}

\subsection{个人项目}

\tllabelcventry{2012}{0}{2012.6}{实现基于用户和物品的协同过滤算法}{}{}{}{
\itshape实现了基于用户和物品的协同过滤算法,并使用GroupLens提供的MovieLensss数据集进行评测。
\upshape
\begin{itemize}
\item 使用python开发,在100W训练集的情况下,准确度可以达到25\%
\item github: https://github.com/iamccme/recommend-CF
\end{itemize}}

%------------------------------------------------
\tllabelcventry{2013}{0}{2013.5}{基于朴素贝叶斯的文本分类器}{}{}{}{
\itshape实现了朴素贝叶斯文本分类器,并使用搜狗语料库迷你版进行训练。
\upshape
\begin{itemize}
\item 使用C++进行开发,分类准确度可以达到55\%
\item github:https://github.com/iamccme/my-naive-bayes-classifier
\end{itemize}}
%------------------------------------------------
\tllabelcventry{2013}{0}{2013.4-2013.5}{微博的情感分析}{}{}{}{
\itshape与别人合作完成。负责情感词的扩展以及算分。
\upshape
\begin{itemize}
\item 使用JAVA进行开发。训练数据从爬盟下载,分词工具使用NLPIR,机器学习算法模块使用的是weka的接口,微博doc的索引build使用的是lucene的工具
\item github:https://github.com/iamccme/weibo-mining
\end{itemize}}
%----------------------------------------------------------------------------------------
%	COMPUTER SKILLS SECTION
%----------------------------------------------------------------------------------------

\section{专业技能}
 
\footnotetext[1]{主要编程语言/常使用工具,了解相对深入}
\footnotetext[2]{较系统地学习过,了解基本原理}
\footnotetext[3]{曾在项目中使,用了解基本语法/用法}

\subsection{开发技能}
\cvitem{编程语言}{\texttt{Python\footnotemark[1], C\footnotemark[2], C++\footnotemark[1], JAVA\footnotemark[3] ,PHP\footnotemark[3],  Shell/Bash\footnotemark[2], Ruby\footnotemark[2]}}
\cvitem{开发框架}{\texttt{django\footnotemark[2], rails\footnotemark[3]}}
\cvitem{代码管理}{\texttt{Git\footnotemark[2], SVN\footnotemark[1]}}
\cvitem{Web开发}{\texttt{HTML/XHTML\footnotemark[2], CSS\footnotemark[3]}}
\cvitem{数据库}{\texttt{MySQL\footnotemark[2]}}
\cvitem{工具}{\texttt{Hive\footnotemark[2], Hudson\footnotemark[2]}}
\cvitem{文本编辑}{\LaTeX\footnotemark[3], \texttt{Office\footnotemark[2], Vim\footnotemark[2]}}
\cvitem{操作系统}{\texttt{GNU/Linux(Ubuntu/RHEL/Arch)\footnotemark[1], OSX\footnotemark[2], Windows\footnotemark[2]}}
\cvitem{Web服务器}{\texttt{Apache\footnotemark[3], lighttpd\footnotemark[2]}}

%----------------------------------------------------------------------------------------
%	AWARDS SECTION
%----------------------------------------------------------------------------------------

\section{所获奖项}

\subsection{校内荣誉}
\cvitem{2013}{北京邮电大学 \textbf{一等奖学金}}
\cvitem{2011, 2012}{北京邮电大学 \textbf{二等奖学金}}
\cvitem{2010}{北京邮电大学 \textbf{优秀团员}}
\cvitem{2009}{北京邮电大学 \textbf{优秀学生干部}}



%----------------------------------------------------------------------------------------
%	LANGUAGES SECTION
%----------------------------------------------------------------------------------------

\section{语言能力}

\cvitemwithcomment{英语}{CET6 -- 439}{}
\cvitemwithcomment{英语}{CET4 -- 514}{}

%----------------------------------------------------------------------------------------
%	INTERESTS SECTION
%----------------------------------------------------------------------------------------

%----------------------------------------------------------------------------------------
%	COVER LETTER
%----------------------------------------------------------------------------------------

% To remove the cover letter, comment out this entire block

% \clearpage

% \recipient{HR Departmnet}{Corporation\\123 Pleasant Lane\\12345 City, State} % Letter recipient
% \date{\today} % Letter date
% \opening{Dear Sir or Madam,} % Opening greeting
% \closing{Sincerely yours,} % Closing phrase
% \enclosure[Attached]{curriculum vit\ae{}} % List of enclosed documents

% \makelettertitle % Print letter title

% \lipsum[1-2] % Dummy text

% \makeletterclosing % Print letter signature

%----------------------------------------------------------------------------------------

%\end{CJK}
\end{document}
